\section{Pruning Neurons to Shrink Neural Networks}\label{sec2}
As discussed in Section \ref{sec1} our aim is to leverage the highly non-uniform distribution of the learning representation in pre-trained neural networks to eliminate redundant neurons, without focusing on individual weight parameters. Taking this approach enables us to remove all the weights (incoming and outgoing) associated with a non-contributing neuron at once. We would like to note here that in an ideal scenario, based on the neuron interdependency theory put forward by \cite{mozer1989skeletonization}, one would evaluate all possible combinations of neurons to remove (one at a time, two at a time, three at a time and so forth) to find the optimal subset of neurons to keep. This is computationally unacceptable, and so we will only focus on removing one neuron at a time and explore more ``greedy'' algorithms to do this in a more efficient manner.

The general approach taken to prune an optimally trained neural network here is to create a ranked list of all the neurons in the network based off of one of the 3 proposed ranking criteria: a brute force approximation, a linear approximation and a quadratic approximation of the neuron's impact on the output of the network. We then test the effects of removing neurons on the accuracy and error of the network. All the algorithms and methods presented here are easily parallelizable as well.

One last thing to note here before moving forward is that the methods discussed in this section involve some non-trivial derivations which are beyond the scope of this paper. We are more focused on analyzing the implications of these methods on our understanding of neural network learning representations. However, a complete step-by-step derivation and proof of all the results presented is provided in the Supplementary Material as an Appendix.

\subsection{Brute Force Removal Approach}
This is perhaps the most naive yet the most accurate method for pruning the network. It is also the slowest and hence possibly unusable on large-scale neural networks with thousands of neurons. This method explicitly evaluates each neuron in the network. The idea is to manually check the effect of every single neuron on the output. This is done by running a forward propagation on the validation set $K$ times (where $K$ is the total number of neurons in the network), turning off exactly one neuron each time (keeping all other neurons active) and noting down the change in error. Turning a neuron off can be achieved by simply setting its output to 0. This results in all the outgoing weights from that neuron being turned off. This change in error is then used to generate the ranked list. 

\input{taylor_series.tex}

\subsubsection{Linear Approximation Approach}
We can use equation \ref{eq:ts3} to get the linear error approximation of the change in error due to the $k$th neuron being turned off and represent it as $\Delta E_{k}^1$ as follows:

\begin{align}
\Delta E_{k}^1 = - O_k\cdot \left.\pdv{E}{O}\right|_{O_k},
\end{align}

The derivative term above is the first-order gradient which represents the change in error with respect to the output a given neuron. This term can be collected during back-propagation. As we shall see further in this section, linear approximations are not reliable indicators of change in error but they provide us with an interesting basis for comparison with the other methods discussed in this paper.

\subsubsection{Quadratic Approximation Approach}
As above, we can use equation \ref{eq:ts3} to get the quadratic error approximation of the change in error due to the $k$th neuron being turned off and represent it as $\Delta E_{k}^2$ as follows:

\begin{align}
\Delta E_{k}^2 =  - O_k\cdot \left.\pdv{E}{O}\right|_{O_k} + 0.5\cdot O_k^2\cdot \left.\pdv[2]{E}{O}\right|_{O_k},
\end{align}

The additional second-order gradient term appearing above represents the quadratic change in error with respect to the output of a given neuron. This term can be generated by performing back-propagation using second order derivatives. Collecting these quadratic gradients involves some non-trivial mathematics, the entire step-by-step derivation procedure of which is provided in the Supplementary Material as an Appendix.
\input{pruning_algorithm.tex}